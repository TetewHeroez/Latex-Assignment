\documentclass[10pt,openany,a4paper]{article}
\usepackage{graphicx} 
\usepackage{multirow}
\usepackage{enumitem}
\usepackage{amssymb}
\usepackage{amsmath}
\usepackage{xcolor}
\usepackage{cancel}
\usepackage{tcolorbox}
\usepackage{geometry}
\usepackage{tikz}
	\geometry{
		total = {160mm, 237mm},
		left = 25mm,
		right = 35mm,
		top = 30mm,
		bottom = 30mm,
	}
\newcommand{\jawab}{\textbf{Jawab}:}

\begin{document}
    \pagenumbering{gobble}
    \begin{tabular}{|lcl|}
     \hline
     Nama&:&Teosofi Hidayah Agung\\
     NRP&:&5002221132\\
     \hline
    \end{tabular}
    \begin{enumerate}
        \item $U_{n+2}-3U_{n+1}+2U_n=0$\\
        \jawab
        \begin{flalign*}
            U_{n+2}-3U_{n+1}+2U_n&=0&\\
            E^2U_n-3EU_n+2U_n&=0&\\
            (E^2-3E+2)U_n&=0
        \end{flalign*}
        Subtitusi $U_n=p^n$, didapatkan persamaan
        \begin{flalign*}
            p^2-3p+2&=0&\\
            (p-2)(p-1)&=0&\\
            p=2\vee p&=1
        \end{flalign*}
        $\therefore$ PUPB: $U_n=C_1+C_2(2^n)$
        \item[5.] $y_{n+2}-7y_{n+1}+12y_n=0$\\
        \jawab
        \begin{flalign*}
            y_{n+2}-7y_{n+1}+12y_n&=0&\\
            E^2y_n-7Ey_n+12y_n&=0&\\
            (E^2-7E+12)y_n&=0
        \end{flalign*}
        Subtitusi $y_n=p^n$, didapatkan persamaan
        \begin{flalign*}
            p^2-7p+12&=0&\\
            (p-3)(p-4)&=0&\\
            p=3\vee p&=4
        \end{flalign*}
        $\therefore$ PUPB: $y_n=C_1(3^n)+C_2(4^n)$
        \item[7.] $\Delta^2y_n-7\Delta y_n+12y_n=0$\\
        \jawab
        \begin{flalign*}
            \Delta^2y_n-7\Delta y_n+12y_n=0&=0&\\
            (\Delta^2-7\Delta+12)y_n&=0&\\
            (\Delta-3)(\Delta-4)y_n&=0
        \end{flalign*}
        Subtitusi $y_n=p^n$ dan $\Delta=E-1$, didapatkan persamaan
        \begin{flalign*}
            (p-4)(p-5)&=0&\\
            p=4\vee p&=5
        \end{flalign*}
        $\therefore$ PUPB: $y_n=C_1(4^n)+C_2(5^n)$
        \item[15.] $y_{n+2}-3y_{n+1}+2y_n=2^n$\\
        \jawab
        \begin{flalign*}
            y_c&= C_1+C_2(2^n)&\\
            y_p&=\frac{1}{(E-2)(E-1)}2^n=\frac{1}{E-2}\frac{2^n}{2-1}=\frac{1}{E-2}2^n=2^n\frac{1}{2E-2}1&\\
            &=2^n\frac{1}{2(\Delta+1)-2}1=2^{n-1}\frac{1}{\Delta}1=n2^{n-1}
        \end{flalign*}
        $\therefore$ PUPB: $y_n=y_c+y_p=C_1+C_2(2^n)+n2^{n-1}$
        \item[22.] $(\Delta^2+\Delta+1)U_n=n^2$\\
        \jawab\\
        Solusi komplementer:
        \begin{flalign*}
            y_c&: \Delta^2+\Delta+1=(E-1)^2+(E-1)+1=E^2-E+1&\\
            &\Rightarrow p^2-p+1=0&\\
            &\Rightarrow p=\frac{1\pm\sqrt{1^2-4(1)(1)}}{2(1)}=\frac{1\pm\sqrt{-3}}{2}=\frac{1\pm\sqrt{3}i}{2}
        \end{flalign*}
        Karena akar bernilai kompleks dengan $r=|p|=\sqrt{\left(\dfrac{1}{2}\right)^2+\left(\dfrac{\sqrt{3}}{2}\right)^2}=1$ dan $\theta=\tan^{-1}\left(\dfrac{\sqrt{3}/2}{1/2}\right)=\tan^{-1}(\sqrt{3})=\dfrac{\pi}{3}$, diperoleh
        \begin{flalign*}
            y_c&=r^n(C_1\cos(n\theta)+C_2\sin(n\theta))&\\
            y_c&=\boxed{C_1\cos\left(\frac{n\pi}{3}\right)+C_2\sin\left(\frac{n\pi}{3}\right)}
        \end{flalign*}
        Solusi partikulir:
        \begin{flalign*}
            y_p=\frac{1}{\Delta^2+\Delta+1}n^2&=(1-\Delta+\Delta^3-\Delta^4+...)(n^{(2)}+n^{(1)})&\\
            &=(1-\Delta)(n^{(2)}+n^{(1)})&\\
            &=n^{(2)}+n^{(1)}-\Delta(n^{(2)}+n^{(1)})&\\
            &=n^2-(2n^{(1)}+1)&\\
            &=\boxed{n^2-2n-1}
        \end{flalign*}
        $\therefore$ PUPB: $y_n=y_p+y_c=\boxed{n^2-2n-1+C_1\cos\left(\frac{n\pi}{3}\right)+C_2\sin\left(\frac{n\pi}{3}\right)}$

        \item[4.] \[f_n=\{2,3^2,3^2,4^2,...\}\]
        \jawab\\
        $U_n=n^2, n=1,2,3,..$
        \begin{flalign*}
            Z\{n^2\}=\sum_{0}^{\infty}(n+1)^2\,z^{-n}=
        \end{flalign*}
        \item[] 
    \end{enumerate}
\end{document}