\documentclass{article}
\usepackage{graphicx} 
\usepackage{multirow}
\usepackage{enumitem}
\usepackage{amssymb}
\usepackage{amsmath}
\usepackage{xcolor}
\usepackage{cancel}
\usepackage{geometry}
    \geometry{
        paperwidth=23cm, 
        paperheight=30cm
        }
\newcommand{\jawab}{\textbf{Jawab}:}
\newcommand{\R}{\mathbb{R}}
\newcommand{\N}{\mathbb{N}}
\begin{document}
    \pagenumbering{gobble}
    \begin{tabular}{|lcl|}
     \hline
     Nama&:&Teosofi Hidayah Agung\\
     NRP&:&5002221132\\
     \hline
    \end{tabular}
    \begin{enumerate}
        \item Perhatikan barisan fungsi $(f_n)$ yang didefinisikan dengan $f_n(x)=\dfrac{nx}{1+nx^2}$ untuk $x\in A:=[0,\infty)$.
        \begin{enumerate}
            \item Tunjukkan bahwa $(f_n)$ terbatas pada $A$ untuk semua $n\in\N$.\\
            \jawab\\
            Kita perhatikan bahwa $f_n(x)=\dfrac{nx}{1+nx^2}$. Karena $x\geq 0$ dan $n\in\N$, maka $nx\geq 0$ dan $1+nx^2\geq 1$. Sehingga $f_n(x)\leq \dfrac{nx}{1}$. Dengan demikian, $f_n(x)$ terbatas pada $A$ untuk semua $n\in\N$.
            \item Tunjukkan bahwa $(f_n)$ konvergen titik-demi-titik ke suatu fungsi $f$, tetapi tidak terbatas.
            \item Apakah $(f_n)$ konvergen seragam pada $A$? Jelaskan!
        \end{enumerate}
        \item Jika $\sum a_n$ konvergen mutlak dan $(b_n)$ barisan terbatas, tunjukkan bahwa $\sum a_nb_n$ konvergen mutlak.
        \item Tunjukkan bahwa deret $\dfrac{1}{1^2}+\dfrac{1}{2^3}+\dfrac{1}{3^2}+\dfrac{1}{4^3}+\dots$ adalah konvergen,tetapi uji rasio dan uji akar gagal \\~\\
        diterapkan untuk memeriksa konvergensi deret tersebut.
        \item Diberikan $\sum a_n$ deret yang konvergen mutlak. Tunjukkan bahwa $\sum a_n \sin(nx)$ adalah deret yang konvergen mutlak dan seragam.
    \end{enumerate}
\end{document}
