\documentclass[10pt,openany,a4paper]{article}
\usepackage{graphicx} 
\usepackage{multirow}
\usepackage{enumitem}
\usepackage{amssymb}
\usepackage{amsmath}
\usepackage{xcolor}
\usepackage{cancel}
\usepackage{tcolorbox}
\usepackage{geometry}
\usepackage{tikz}
	\geometry{
		total = {160mm, 237mm},
		left = 25mm,
		right = 35mm,
		top = 30mm,
		bottom = 30mm,
	}
\newcommand{\jawab}{\textbf{Jawab}:}

\begin{document}
    \pagenumbering{gobble}
    \begin{tabular}{|lcl|}
     \hline
     Nama&:&Teosofi Hidayah Agung\\
     NRP&:&5002221132\\
     \hline
    \end{tabular}
    \begin{enumerate}
        \item Cari $P_4(x)$ dan $P_5(x)$.\\
        \jawab\\
        Rumus rekursif Legendre adalah $P_{n+1}(x) = \dfrac{2n+1}{n+1}xP_n(x) - \dfrac{n}{n+1}P_{n-1}(x)$. Dengan diketahui $P_2(x) = \dfrac{1}{2}(3x^2-1)$ dan $P_3(x) = \dfrac{1}{2}(5x^3-3x)$.
        \begin{flalign*}
            \Rightarrow P_{4}(x) &= \dfrac{2(3)+1}{3+1}xP_3(x) - \dfrac{3}{3+1}P_{2}(x)&\\
            &= \dfrac{7}{4}x\left(\dfrac{1}{2}(5x^3-3x)\right) - \dfrac{3}{4}\left(\dfrac{1}{2}(3x^2-1)\right)&\\
            &= \dfrac{7}{8}\left(5x^4-3x^2\right) - \dfrac{3}{8}\left(3x^2-1\right)&\\
            &= \dfrac{35}{8}x^4 - \dfrac{21}{8}x^2 - \dfrac{9}{8}x^2 + \dfrac{3}{8}&\\
            &= \dfrac{35}{8}x^4 - \dfrac{30}{8}x^2 + \dfrac{3}{8}&\\
            &= \frac{1}{8}\left(35x^4-30x^2+3\right)&\\
            \Rightarrow P_{5}(x) &= \dfrac{2(4)+1}{4+1}xP_4(x) - \dfrac{4}{4+1}P_{3}(x)&\\
            &= \dfrac{9}{5}x\left(\dfrac{1}{8}\left(35x^4-30x^2+3\right)\right) - \dfrac{4}{5}\left(\dfrac{1}{2}(5x^3-3x)\right)&\\
            &= \dfrac{9}{40}\left(35x^5-30x^3+3x\right) - \dfrac{4}{5}\left(\dfrac{1}{2}(5x^3-3x)\right)&\\
            &= \frac{1}{8}\left(63x^5-70x^3+15x\right)
        \end{flalign*}
        \item Buktikan $(1+x^2)P_n'(x)=n[P_{n-1}(x)-xP_n(x)]$.\\
        \jawab\\
        Diketahui sifat 12 yaitu
        \begin{flalign*}
            nP_n(x) &= xP_n'(x)-P_{n-1}'(x)&\\
            nxP_n(x) &= x^2P_n'(x) - xP_{n-1}'(x)\hdots(1)
        \end{flalign*}
        Disisi lain
        \begin{flalign*}
            (n+1)P_n(x)&=P_{n+1}'(x)-xP_{n}'(x)&\\
            nP_{n-1}(x)&=P_n'(x)-xP_{n-1}'(x)&\\
            xP_{n-1}(x)&=P_n'(x)-P_{n-1}(x)\hdots(2)
        \end{flalign*}
        Subtitusi persamaan (2) ke persamaan (1) didapat
        \[n[P_{n-1}(x)-xP_n(x)] = nxP_n(x) = (1-x^2)P_n'(x)\]
        \item Kontruksi polinomial Legendre dari $f(x)=x^4+2x^3+2x^2-x+3$.\\
        \jawab\\
        \begin{itemize}
            \item Koefisien $x^4$.
            \begin{flalign*}
                \frac{8}{35}P_4(x)&=\frac{8}{35}\left(\frac{1}{8}\left(35x^4-30x^2+3\right)\right)&\\
                &=x^4-\frac{30}{35}x^2+\frac{3}{35}=x^4-\frac{6}{7}x^2+\frac{3}{35}
            \end{flalign*}
            \item Koefisien $2x^3$.
            \begin{flalign*}
                \frac{4}{5}P_3(x)&=\frac{4}{5}\left(\frac{1}{2}(5x^3-3x)\right)&\\
                &=2x^3-\frac{6}{5}x
            \end{flalign*}
            \item Koefisien $2x^2$.
            Pertimbangkan Koefisien $x^2$ pada $P_4(x)$.
            \begin{flalign*}
                -\frac{6}{7}x^2+\frac{3k}{2}x^2=2x^2&\Rightarrow k=\frac{40}{21}
            \end{flalign*}
            \item Koefisien $-x$.
            \begin{flalign*}
                -\frac{6}{5}x+kx=-x&\Rightarrow k=\frac{1}{5}
            \end{flalign*}
            \item Konstanta $3$
            \begin{flalign*}
                \frac{3}{35}-\frac{20}{21}k=3&\Rightarrow k=\frac{58}{15}
            \end{flalign*}
        \end{itemize}
        $\displaystyle\therefore f(x)=\frac{8}{35}P_4(x)+\frac{4}{5}P_3(x)-\frac{40}{21}P_4(x)+\frac{1}{5}P_3(x)+\frac{58}{15}P_4(x)$ 
        \item 
    \end{enumerate}
\end{document}
