\documentclass{article}
\usepackage{graphicx} 
\usepackage{multirow}
\usepackage{enumitem}
\usepackage{amssymb}
\usepackage{amsmath}
\usepackage{xcolor}
\usepackage{cancel}
\usepackage{tcolorbox}
\usepackage{geometry}
\usepackage{tikz, pgfplots, tkz-euclide,calc}
    \usetikzlibrary{patterns,snakes,shapes.arrows,3d}
	\geometry{
		total = {160mm, 237mm},
		left = 25mm,
		right = 35mm,
		top = 30mm,
		bottom = 30mm,
	}

\newcommand{\jawab}{\textbf{Jawab}:}
\begin{document}
    \pagenumbering{gobble}
    \begin{tabular}{|lcl|}
     \hline
     Nama&:&Teosofi Hidayah Agung\\
     NRP&:&5002221132\\
     \hline
    \end{tabular}
    \begin{enumerate}
        \item Buktikan jarak titik $P_0(x_0,y_0,z_0)$ ke bidang $ax+by+cz+d=0$ adalah
        \[\textrm{d}=\frac{|ax_0+by_0+cz_0+d}{\sqrt{a^2+b^2+c^2}}\]
        Hitung jarak antara bidang $x+2y-2z=3$ dan bidang $2x+4y-4z=7$.\\
        \jawab\\
        Misalkan $P_1(x_1,y_1,z_1)$ adalah titik yang terdapat pada bidang $ax+by+cz+d=0$, maka 
        didapat
        \begin{align}
            ax_1+by_1+cz_1+d=0
        \end{align}
        Kemudian kita bisa paksa titik $P_0$ terhadap bidang tersebut sehingga menjadi
        \begin{align}
            ax_0+by_0+cz_0+d=ax_0+by_0+cz_0+d
        \end{align}
        Eliminasi persamaan (1) dan (2)
        \[\frac
        {
            \!\begin{aligned} 
                ax_0+by_0+cz_0+d&=ax_0+by_0+cz_0+d\\ 
                ax_1+by_1+cz_1+d&=0
            \end{aligned}
        }
        {
            a(x_0-x_1)+b(y_0-y_1)+c(z_0-z_1)=ax_0+by_0+cz_0+d
        }
        \ -\]
        Menggunakan definisi perkalian titik, didapatkan
        \begin{align*}
            (a,b,c)\cdot(x_0-x_1,y_0-y_1,z_0-z_1)&=ax_0+by_0+cz_0+d\\
            ||a,b,c||\cdot||x_0-x_1,y_0-y_1,z_0-z_1||\cdot\cos(\theta)&=ax_0+by_0+cz_0+d\\
            \left(\sqrt{a^2+b^2+c^2}\right)\left(\sqrt{(x_0-x_1)^2+(y_0-y_1)^2+(z_0-z_1)^2}\right)\cos(\theta)&=ax_0+by_0+cz_0+d\\
            \left(\sqrt{(x_0-x_1)^2+(y_0-y_1)^2+(z_0-z_1)^2}\cos(\theta)\right)&=\frac{ax_0+by_0+cz_0+d}{\sqrt{a^2+b^2+c^2}}
        \end{align*}
        Ingat bahwa vektor normal bidang $ax+by+cz+d=0$ adalah $N=(a,b,c)$. Perhatikan juga 
        bahwa vektor normal sejajar dengan vektor yang melewati titik $P_0$ dan $P_1$ sehingga 
        sudut yang terbentuk $\theta=\{0,\pi\}$. Sehingga jika kita nilai mutlak kedua ruas, 
        maka terbukti rumus diatas
        \[\textrm{d}=\sqrt{(x_0-x_1)^2+(y_0-y_1)^2+(z_0-z_1)^2}=\frac{|ax_0+by_0+cz_0+d|}{\sqrt{a^2+b^2+c^2}}\] 
        Sekarang untuk menghitung jarak antar bidang $x+2y-2z=3$ dan bidang $2x+4x-4z=7$, hanya 
        cukup perlukan sebuah titik disalah satu bidang kemudian tinjau jarak antar titik tersebut
        ke bidang lain.\\
        Ambil titik $(3,0,0)$ yang terdapat pada $x+2y-2z=3$ kemudian tinjau jaraknya ke bidang 
        $2x+4x-4z-7=0$. Kemudian subtitusikan kedalam rumus sebelumnya
        \[\textrm{d}=\frac{|2x_0+4y_0-4z_0-7|}{\sqrt{2^2+4^2+(-2)^2}}=\frac{|2(3)+4(0)-4(0)-7|}{\sqrt{36}}=\frac{|-1|}{6}=\frac{1}{6}\]

        \item Suatu partikel bergerak mengikuti kurva dengan persamaan parametrik 
        $x=2t^2,\,y=t^2-4t,\,z=3t-5$. Pada waktu $t=1$, partikel bergerak searah dengan vektor 
        $\vec{u}=\vec{i}-3\vec{j}+2\vec{k}$. Tentukan laju (besarnya kecepatan) partikel pada 
        waktu $t=1$ (searah vektor $\vec{u}$).\\
        \jawab
        \begin{flalign*}
            r(t)&=x(t)\vec{i}+y(t)\vec{j}+z(t)\vec{k}&\\
            &=2t^2\,\vec{i}+(t^2-4t)\,\vec{j}+(3t-5)\,\vec{k}&\\
            r'(t)&=4t\,\vec{i}+(2t-4)\,\vec{j}+3\,\vec{k}&\\
            r'(1)&=4\,\vec{i}-2\,\vec{j}+3\,\vec{k}
        \end{flalign*}
        Untuk mencari laju vektor singgung terhadap vektor $\vec{u}$ diperlukan panjang proyeksi 
        $r'(1)$ terhadap $\vec{u}$.
        \begin{flalign*}
            \textrm{Laju}&=\frac{r'(1)\cdot\vec{u}}{|\vec{u}|}&\\
            &=\frac{(4)(1)+(-2)(-3)+(3)(2)}{\sqrt{1^2+(-3)^2+2^2}}&\\
            &=\frac{4+6+6}{\sqrt{1+9+4}}&\\
            &=\frac{16}{\sqrt{14}}
        \end{flalign*}

        \item Dapatkan kelengkungan dari kurva $x=t,\,y=4t^{\frac{3}{2}},\,z=-t^2$ di titik 
        $(1,4,-1)$.\\
        \jawab\\
        Rumus kelengkungan salah satunya adalah
        \[\kappa=\frac{|r'(t)\times r''(t)|}{|r'(t)|^3}\]
        Kita perlu mendapatkan turunan pertama dan kedua dari $r(t)$.
        \begin{flalign*}
            r(t)&=x(t)\vec{i}+y(t)\vec{j}+z(t)\vec{k}&\\
            &=t\,\vec{i}+4t^{\frac{3}{2}}\,\vec{j}-t^2\,\vec{k}&\\
            r'(t)&=\vec{i}+6t^{\frac{1}{2}}\,\vec{j}-2t\,\vec{k}&\\
            r''(t)&=3t^{-\frac{1}{2}}\,\vec{j}-2\,\vec{k}
        \end{flalign*}
        Subtitusi $t=1$.
        \begin{flalign*}
            r'(1)&=\vec{i}+6\,\vec{j}-2t\,\vec{k}&\\
            r''(1)&=3\,\vec{j}-2\,\vec{k}&\\
        \end{flalign*}
        Kemudian hitung $|r'(1)\times r''(1)|$ dan $|r'(1)|$.
        \begin{flalign*}
            |r'(1)|&=\sqrt{1^2+6^2+(-2)^2}=\sqrt{41}&\\
            |r'(1)\times r''(1)|&=\left|\begin{vmatrix}
                i&j&k\\
                1&6&-2\\
                0&3&2
            \end{vmatrix}\right|&\\
            &=|6i-2j+3k|&\\
            &=\sqrt{6^2+(-2)^2+3^2}=7
        \end{flalign*}
        $\therefore$ Kelengkungannya adalah $\kappa=\dfrac{|r'(1)\times r''(1)|}{|r'(1)|^3}=\dfrac{7}{\sqrt{41^3}}$

        \item Diketahui permukaan benda $z=f(x,y)$ yang diberikan oleh persamaan parametrik:
        \[\begin{cases}
            x&=u\,\cos(v)\\
            y&=u\,\sin(v)\\
            z&=2u
        \end{cases}\]
        \begin{enumerate}
            \item Tunjukkan bahwa permukaan $z=f(x,y)$ adalah kerucut.\\
            \jawab
            \begin{flalign*}
                \begin{cases}
                    x^2&=u^2\,\cos^2(v)\\
                    y^2&=u^2\,\sin^2(v)\\
                    z^2&=4u^2
                \end{cases}\Longrightarrow
                \begin{cases}
                    x^2+y^2&=u^2\\
                    z^2&=4u^2
                \end{cases}\Longrightarrow z^2=4x^2+4y^2
            \end{flalign*}
            $\therefore$ Terbukti bahwa $z=f(x,y)$ adalah kerucut.
            \item Nyatakan permukaan benda tersebut sebagai fungsi vektor $r(u,v)$.\\
            \jawab
            \begin{flalign*}
                r(u,v)&=x(u,v)\vec{i}+y(u,v)\vec{j}+z(u,v)\vec{k}&\\
                &=u\,\cos(v)\,\vec{i}+u\,\sin(v)\,\vec{j}-2u\,\vec{k}
            \end{flalign*}
            \item Dapatkan vektor normal permukaan $z=f(x,y)$ pada saat $(u,v)=(2,\dfrac{\pi}{4})$.\\
            \jawab\\
            Misalkan $\phi=z^2-4x^2-4y^2$ lalu dengan menggunakan rumus operator gradien didapatkan
            \begin{flalign*}
                \nabla\phi&=\left(\frac{\partial}{\partial x}+\frac{\partial}{\partial y}+\frac{\partial}{\partial z}\right)\phi&\\
                &=\left(\frac{\partial}{\partial x}+\frac{\partial}{\partial y}+\frac{\partial}{\partial z}\right)(z^2-4x^2-4y^2)&\\
                &=-8x~\vec{i}-8y~\vec{j}+2z~\vec{k}&\\
                &=-8u\,\cos(v)~\vec{i}-8u\,\sin(v)~\vec{j}+4u~\vec{k}&\\
                |\nabla\phi|&=\sqrt{(-8x)^2+(-8y)^2+(2z)^2}&\\
                &=\sqrt{64x^2+64y^2+4z^2}&\\
                &=2\sqrt{16u^2\,\cos^2(v)+16u^2\,\sin^2(v)+4u^2}&\\
                &=2\sqrt{20u^2}=4\sqrt{5u^2}
            \end{flalign*}
            Subtitusi ke rumus vektor normal
            \[N=\left.\frac{\nabla\phi}{|\nabla\phi|}\right|_{\footnotesize{\begin{matrix}u=2\\v=\frac{\pi}{4}\end{matrix}}}=\frac{1}{8\sqrt{5}}(-8\sqrt{2}~\vec{i}-8\sqrt{2}~\vec{j}+8~\vec{k})=-\frac{\sqrt{2}}{\sqrt{5}}~\vec{i}-\frac{\sqrt{2}}{\sqrt{5}}~\vec{j}+\frac{1}{\sqrt{5}}~\vec{k}\]
            \item Dapatkan persamaan garis normal dan bidang singgung permukaan benda tersebut 
            pada saat $(u,v)=\left(2,\dfrac{\pi}{4}\right)$.\\
            \jawab\\
            Rumus persamaan garis normal
            \[(R-r)=kN\]
            Sehingga persamaan garis normalnya
            \[\frac{-\sqrt{5}(x-2)}{\sqrt{2}}=\frac{-\sqrt{5}(y-2)}{\sqrt{2}}=\frac{\sqrt{5}(z-4)}{\sqrt{1}}\]
            Sedangkan rumus persamaan bidang singgung
            \[(R-r)\cdot N=0\]
            Akibatnya didapatkan persamaan bidangnya
            \begin{flalign*}
                \left(x-\sqrt{2},y-\sqrt{2},z-4\right)\cdot\left(\frac{-\sqrt{2}}{\sqrt{5}},\frac{-\sqrt{2}}{\sqrt{5}},\frac{1}{\sqrt{5}}\right)&=0&\\
                x\sqrt{2}+y\sqrt{2}-z&=0
            \end{flalign*}
        \end{enumerate}
    \end{enumerate}
\end{document}