\documentclass[10pt,openany,a4paper]{article}
\usepackage{graphicx} 
\usepackage{multirow}
\usepackage{enumitem}
\usepackage{amssymb}
\usepackage{amsmath}
\usepackage{xcolor}
\usepackage{geometry}
	\geometry{
		total = {160mm, 237mm},
		left = 25mm,
		right = 35mm,
		top = 30mm,
		bottom = 30mm,
	}

 \begin{document}
    \pagenumbering{gobble}
    \setcounter{section}{3}
    \setcounter{subsection}{1}
    \begin{tabular}{|lcl|}
     \hline
     Nama&:&Teosofi Hidayah Agung\\
     NRP&:&5002221132\\
     \hline
    \end{tabular}
    \subsection{}
    \begin{enumerate}
        \setcounter{enumi}{3}
        \item Tunjukkan jika $X$ dan $Y$ suatu barisan yang sedemikian sehingga $X$ konvergen ke $x\neq0$ dan $XY$ konvergen, maka $Y$ juga konvergen.\\~\\
        \textbf{Jawab}:\\
        Karena $XY=(x_ny_n)$ barisan konvergen, maka dapat diandaikan lim$(x_ny_n)=z$ dan lim$(x_n)=x\neq0$.\\
        Selanjutnya perhatikan bahwa bahwa limit barisan $Y$ adalah lim$(y_n)=\textrm{lim}(\frac{x_ny_n}{x_n})$.
        Dapat disimpulkan bahwa barisan $Y$ konvergen ke $\frac{z}{x}$.
        
    \end{enumerate}
    
    \subsection{}
    \begin{enumerate}
        \setcounter{enumi}{6}
        \item Misal $x_1:=a>0$ dan $x_{n+1}=x_n+1/x_n$ untuk $n\in\mathbb{N}$. Tentukan apakah $(x_n)$ konvergen atau divergen.\\~\\
        \textbf{Jawab}:\\
        Andaikan bahwa $(x_n)$ konvergen ke L, sehingga lim$(x_n)=L$. Menurut teorema ketunggalan limit diperoleh
        \begin{flalign*}
            \textrm{lim}(x_{n})&=\textrm{lim}(x_{n+1})&\\
            \textrm{lim}(x_{n})&=\textrm{lim}(x_n+1/x_n)&\\
            \textrm{lim}(x_{n})&=\textrm{lim}(x_n)+\textrm{lim}(1/x_n)&\\
            L&=L+1/L&\\
            0&=1/L\quad\textrm{\textbf{(Kontradiksi)}}
        \end{flalign*}
        Jadi harusnlah $(x_n)$ divergen.
    \end{enumerate}

    \subsection{}
    \begin{enumerate}
        \setcounter{enumi}{8}
        \item Misalkan untuk setiap subbarisan dari $X=(x_n)$ memiliki subbarisan yang konvergen ke $0$. Tunjukkan bahwa lim $X=0$.\\~\\
        \textbf{Jawab}:\\
        Andaikan lim$(x_n)=L\neq0$, sehingga untuk setiap $\epsilon>0$ terdapat $K(\epsilon)\in\mathbb{N}$ sedemikian sehingga $\forall n\geq K(\epsilon)\in\mathbb{N}$ berlaku $|x_n-L|<\epsilon$.\\
        Menurut sebuah teorema, Jika $X=(x_n)$ konvergen ke $L$ maka sebarang barisan bagian dari $X$ juga konvergen ke $L\neq0$. Namun hal ini kontradiksi bahwa setiap subbarisan dari $X=(x_n)$ memiliki subbarisan yang konvergen ke $0$.\\~\\
        $\therefore$ dapat disimpulkan barisan $X=(x_n)$ konvergen ke $0$ atau lim $X=0$
    \end{enumerate}

    \subsection{}
    \begin{enumerate}
        \setcounter{enumi}{3}
        \item Tunjukkan menurut definisi bahwa jika $(x_n)$ dan $(y_n)$ adalah barisan Cauchy, maka $(x_n+y_n)$ dan $(x_ny_n)$ juga merupakan barisan Cauchy.\\~\\
        \textbf{Jawab}:\\
        Menurut definisi, barisan $A=(a_n)$ dikatakan barisan Cauchy jika untuk setiap $\epsilon>0$, $\exists H(\epsilon)\in\mathbb{N}$ sedemikian sehingga $\forall n,m \geq H(\epsilon)$ berlaku $|a_n-a_m|<\epsilon$.
        \begin{enumerate}[label=(\roman*)]
            \item Untuk $(x_n+y_n)$ diberikan $\forall n,m \geq H(\epsilon)$, maka diperoleh
            \begin{flalign*}
                |(x_n+y_n)-(x_m+y_m)|&=|(x_n-x_m)+(y_n-y_m)|&\\
                &\leq|x_n-x_m|+|y_n-y_m|&\\
            \end{flalign*}
            Karena $(x_n)$ dan $(y_n)$ adalah barisan Cauchy, maka 
            \begin{flalign*}
                \forall n,m \geq H(\epsilon)_x &\Longrightarrow |x_n-x_m|<\epsilon/2&\\
                &\textrm{dan}&\\
                \forall n,m \geq H(\epsilon)_y &\Longrightarrow |y_n-y_m|<\epsilon/2&\\
            \end{flalign*}
            Sehingga untuk $H(\epsilon)=\textrm{sup}\{H(\epsilon)_x,H(\epsilon)_y\}$ diperoleh
            \begin{flalign*}
                |x_n-x_m|+|y_n-y_m|<\epsilon/2+\epsilon/2=\epsilon
            \end{flalign*}
            Maka didapat $|(x_n+y_n)-(x_m+y_m)|<\epsilon$ yang dimana merupakan definisi bahwa $(x_n+y_n)$ adalah barisan Cauchy\\

            \item Untuk $(x_ny_n)$ diberikan $\forall n,m \geq H(\epsilon)$, maka diperoleh
            \begin{flalign*}
                |x_ny_n-x_my_m|&=|x_ny_n-x_my_n+x_my_n-x_my_m|&\\
                &\leq|x_ny_n-x_my_n|+|x_my_n-x_my_m|&\\
                &\leq|y_n||x_n-x_m|+|x_m||y_n-y_m|
            \end{flalign*}
            Sebuah teorema mengatakan \textit{"jika barisan $(a_n)$ konvergen maka barisan itu terbatas"}. Karena $(x_n)$ dan $(y_n)$ adalah barisan Cauchy, maka sudah tentu barisan itu konvergen juga terbatas. Sehingga didapat $|x_m|\leq M_x$ dan $|y_n|\leq M_y$, lalu dapat diambil $M=\textrm{sup}\{M_x,M_y\}$ yang berakibat
            \begin{flalign*}
                |y_n||x_n-x_m|+|x_m||y_n-y_m|\leq M|x_n-x_m|+M|y_n-y_m|
            \end{flalign*}
            Kembali lagi bahwa $(x_n)$ dan $(y_n)$ adalah barisan Cauchy, maka
            \begin{flalign*}
                \forall n,m \geq H(\epsilon)_x &\Longrightarrow |x_n-x_m|<\frac{\epsilon}{2M}&\\
                &\textrm{dan}&\\
                \forall n,m \geq H(\epsilon)_y &\Longrightarrow |y_n-y_m|<\frac{\epsilon}{2M}&\\
            \end{flalign*}
            Sehingga untuk $H(\epsilon)=\textrm{sup}\{H(\epsilon)_x,H(\epsilon)_y\}$ diperoleh
            \begin{flalign*}
                M|x_n-x_m|+M|y_n-y_m|< M\left(\frac{\epsilon}{2M}\right)+M\left(\frac{\epsilon}{2M}\right)=\epsilon
            \end{flalign*}
             Maka didapat $|(x_ny_n)-(x_my_m)|<\epsilon$ yang dimana merupakan definisi bahwa $(x_ny_n)$ adalah barisan Cauchy\\
        \end{enumerate}
    \end{enumerate}
 \end{document}